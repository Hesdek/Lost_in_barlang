\documentclass{article}
\usepackage[utf8]{inputenc}
\usepackage[spanish]{babel}
\usepackage{listings}
\usepackage{graphicx}
\graphicspath{ {images/} }
\usepackage{cite}

\begin{document}

\begin{titlepage}
    \begin{center}
        \vspace*{1cm}
            
        \Huge
        \textbf{Descripción del proyecto}
            
        \vspace{0.5cm}
        \LARGE
        Informática II
            
        \vspace{1.7cm}
            
        \textbf{Daniel Pérez Gallego}
        
        \vspace{0.3cm}
        
        \textbf{Jorge Andrés Montaña Cisneros}
            
        \vfill
            
        \vspace{0.8cm}
            
        \Large
        Despartamento de Ingeniería Electrónica y Telecomunicaciones\\
        Universidad de Antioquia\\
        Medellín\\
        Octubre de 2021
            
    \end{center}
\end{titlepage}

\tableofcontents

\section{Introducción}
Decidimos realizar un juego de plataformas en 2D, con el objetivo principal de que el jugador analice el escenario y comprenda las diferentes estrategias que requiere cada enemigo para vencerlo y que finalmente haga uso del mapa para escapar o acabar con los monstruos. Todo esto con la finalidad de aplicar todo lo aprendido en este semestre, ya que consideramos que un juego de este tipo es lo suficientemente completo para demostrar todo lo aprendido en el transcurso de la asignatura informática.

\section{Historia del juego}
Toda la humanidad ha centrado su atención en una isla remota en el océano pacífico, se creyó por siglos que era un volcán durmiente, pero en realidad era un conjunto de cuevas gigantes dentro de una montaña, presuntamente más profundas que la fosa de las marianas, un nivel radioactivo que supera a la pata de elefante en Chernobyl, animales gigantes mutados nunca antes vistos y donde la luz del sol jamás llegará.\\

Cuando llegó a la isla, notó una atmósfera tenebrosa, sentía el peligro incluso en las corrientes del aire, sin embargo, esto no detuvo al nuestro aventurero, que, sin temor alguno, se adentró en las profundidades de la cueva, sin saber a qué se enfrentaría…\\

Un peligro que espantó a casi todos los humanos… excepto para el mejor aventurero de la historia. ¡En búsqueda de aún más fama y riquezas, de ser digno de pertenecer a los libros de historia de la humanidad!... Se lanzó a su muerte segura, sin equipo ni ayuda… idiota.\\

\textbf{''[Lo llamaron “Barlang” la cueva dentro de una isla imposible''}]\\


\section{Modalidad de juego}
Desarrollaremos un de plataformas en 2D con obstáculos, enemigos con distintos ataques y estrategias para vencerlos, con la posibilidad de ataques a distancia y cuerpo a cuerpo.\\

Usaremos la clásica fórmula de un juego de plataformas, pero con mecánicas únicas para que el jugador no se aburra de estar “saltando y saltando”. Para ello ideamos enemigos con estrategias diferentes, efectos visuales y efectos secundarios según el enemigo, mapas atractivos y un sistema de ataque a distancia limitado para que el jugador no abuse de este recurso, todo esto con la temática de “Un don nadie en una cueva mortal”.\\

Con todo lo anterior en cuenta, el objetivo es llegar de un punto A al punto B y derrotar a un enemigo final, recolectando recursos para hacer la aventura más amena, cada nivel tendrá varios enemigos con sus posiciones predefinidas, diseñadas para darle un reto al jugador.

\section{Diseño}


\section{Motivación}
Nuestro primer objetivo fue darle un gameplay dinámico e intenso, para que el jugador pueda usar distintas estrategias y caminos para superar los retos que se le presentan, dándole herramientas como el parkour, los ataques a distancia y el mismo mapa.\\

Propusimos darle un arte simple de hacer, pero muy llamativo y único para el jugador. Para ello, acordamos hacer todos los escenarios y personajes como monstruos mutantes, para darle una ambientación más oscura, parecido al juego “Hollow knight”, pero con personajes, ambientación y música que transmitan terror, misterio y osadía.

\section{Problemas del desarrollo}

\bibliographystyle{IEEEtran}
\bibliography{references}

\end{document}